Modern society has an increasingly high demand on data transmission. Much of that data transmission takes place through communication over the frequency spectrum. The channels on the spectrum are limited resources. People realize, at certain times of the day, some channels have been overloaded, and others just have not ever been fully utilized. A spectrum management system may be beneficial to remedy this efficiency issue. Throughout the years, being one of the proposed systems, Cognitive Radio Network (CRN) technology progressed into a wide range of studies including geolocation, data throughput rate, channel handoff selection algorithm, etc, which provide the fundamental support for the spectrum management system. To move the CRN technology forward, in this thesis, we propose a physical, scalable testbed for some of the extant CRN methodologies. This testbed integrates IEEE standards, FCC guidelines, and other TV band regulations to emulate CRN in real time. With careful component selections, we include sufficient operational functionalities in the system, while at the same time making sure it remains affordable. We evaluate the technical feasibility of the testbed by studying different simple CRN logics. When comparing a system with a selection table implemented to those with na\"ive selection methods, there is more than a 60 percent improvement in the overall performance.
