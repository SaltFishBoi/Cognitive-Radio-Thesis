Modern society places an increasingly high demand on data transmission. Much of that data transmission takes place through communication over the frequency spectrum. The channels on the spectrum are limited resources. Researchers realize that at certain times of day some channels are overloaded, while others are not being fully utilized. A spectrum management system may be beneficial to remedy this efficiency issue. One of the proposed systems, Cognitive Radio Network (CRN), has progressed over the years thanks to studies on a wide range of subjects, including geolocation, data throughput rate, and channel handoff selection algorithm, which provide fundamental support for the spectrum management system. To move CRN technology forward, in this thesis we propose a physical, scalable testbed for some of the extant CRN methodologies. This testbed integrates IEEE standards, FCC guidelines, and other TV band regulations to emulate CRN in real time. With careful component selections, we include sufficient operational functionalities in the system, while at the same time making sure it remains affordable. We evaluate the technical feasibility of the testbed by studying several simple CRN logics. When comparing a system with a selection table implemented to those with na\"ive selection methods, there is more than a 60 percent improvement in the overall performance.
